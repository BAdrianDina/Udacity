


\documentclass{beamer}

\usepackage{beamerthemesplit}
\setbeamercovered{invisible}
\usepackage{verbatim}
\usetheme{Frankfurt}
\usepackage{mathtools}

\usepackage{tikz}

\usetikzlibrary{shapes.geometric, arrows, positioning, decorations.pathreplacing}
\tikzstyle{startstop} = [rectangle, rounded corners, minimum width=3cm, minimum height=1cm,text centered, draw=black, fill=red!30]
\tikzstyle{process} = [rectangle, minimum width=3.5cm, minimum height=1cm, text centered, draw=black, fill=orange!30]
\tikzstyle{onchain} = [rectangle, minimum width=3.5cm, minimum height=1cm, text centered, draw=black, fill=blue!30]
\tikzstyle{offchain} = [rectangle, minimum width=3.5cm, minimum height=1cm, text centered, draw=black, fill=green!30]
\tikzstyle{arrow} = [thick,->,>=stealth]



\usetikzlibrary{trees}
\usepackage{smartdiagram}
\usetikzlibrary{calc}
\usetikzlibrary{shapes.geometric}
\usetikzlibrary{arrows.meta}
\usetikzlibrary{positioning}
\usepackage{pgfplots}
\usepackage{pgfmath-xfp,xfp}
\usepackage{pdfpages}
\usepackage[export]{adjustbox}
\usepackage{algorithm2e}
\usepackage{algorithmic}
\usepackage{float}

\setbeamercolor{alerted text}{fg=blue}

\makeatletter
\setbeamertemplate{headline}
{%
  \pgfuseshading{beamer@barshade}%
  \ifbeamer@sb@subsection%
    \vskip-9.75ex%
  \else%
    \vskip-7ex%
  \fi%
  \begin{beamercolorbox}[ignorebg,ht=2.25ex,dp=0.75ex]{section in head/foot}
    \insertsectionnavigationhorizontal{\paperwidth}{}{\hfill\hfill}
  \end{beamercolorbox}%
  \ifbeamer@sb@subsection%
    \begin{beamercolorbox}[ignorebg,ht=2.125ex,dp=1.125ex,%
      leftskip=.3cm,rightskip=.3cm plus1fil]{subsection in head/foot}
      \usebeamerfont{subsection in head/foot}\insertsubsectionhead
    \end{beamercolorbox}%
  \fi%
 \begin{beamercolorbox}[colsep=1.5pt,ht=.75ex]{upper separation line head}
 \end{beamercolorbox}
}%
\makeatother

% Remove navigation symbols:
\setbeamertemplate{navigation symbols}{}

\addtolength{\parskip}{0.5\baselineskip}
\setlength{\arraycolsep}{2pt}

% Font:
% (Currently set to default Computer Modern)
%\usepackage{mathpazo}

% Encoding and languages:
\usepackage[utf8]{inputenc}
%\usepackage[OT2,T1]{fontenc}
\usepackage[ngerman]{babel}

% Standard AMS packages:
\usepackage{amsmath}
\usepackage{amsfonts}
\usepackage{amssymb}
\usepackage{amsthm}
\usepackage{mathrsfs}

% Calligraphic mathcal
\usepackage[mathcal]{euscript}

% Sha support:
\usepackage[OT2,T1]{fontenc}
\DeclareSymbolFont{cyrletters}{OT2}{wncyr}{m}{n}
\DeclareMathSymbol{\Sha}{\mathalpha}{cyrletters}{"58}

% Interior and exterior linking:
\usepackage{url}
\usepackage{hyperref}

% Enumeration and tables:
%\usepackage{enumitem}
\usepackage{multirow}

% Graphics and diagrams:
\usepackage{graphicx}
\usepackage{tikz-cd}
%\usepackage[all]{xy}
% What follows is ugly, but very useful to quickly TeX large diagrams:
\newenvironment{cd2}{\begin{equation*}\begin{tikzcd}}{\end{tikzcd}\end{equation*}}
\newenvironment{cd3}{\begin{equation*}\begin{tikzcd}}{\end{tikzcd}\end{equation*}}
\newenvironment{cd4}{\begin{equation*}\begin{tikzcd}}{\end{tikzcd}\end{equation*}}
\newenvironment{cd5}{\begin{equation*}\begin{tikzcd}}{\end{tikzcd}\end{equation*}}
\newenvironment{ses}{\begin{equation*}\begin{tikzcd}}{\end{tikzcd}\end{equation*}}

% Better if statements:
\usepackage{etoolbox}


% Operator and symbol macros:
\def\defi{\textsf}
\def\ext{\!\mid\!}
\def\limdir{\varinjlim}
\def\liminv{\varprojlim}
\def\eps{\varepsilon}
\def\epsilon{\varepsilon}
\def\theta{\vartheta}
\def\phi{\varphi}
\def\tilde{\widetilde}

% Text macros:
\def\Belyi{Bely\u{\i}}
\def\LiE{\textsc{LiE}}
\def\Magma{\textsc{Magma}}
\def\SageMath{\textsc{SageMath}}
\def\NB{\textbf{N.B.}}
\def\cf{\textit{cf.}}
\def\etal{\textit{et al.}}
\def\etc{\textit{etc.}}
\def\eg{\textit{e.g.}}
\def\ie{\textit{i.e. }}
\def\infra{\textit{infra}}
\def\loccit{\textit{loc.\ cit.}}


% Math operators:
\DeclareMathOperator{\alg}{alg}
\DeclareMathOperator{\Ell}{Ell}
\DeclareMathOperator{\ab}{ab}
\DeclareMathOperator{\ad}{ad}
\DeclareMathOperator{\an}{an}
\DeclareMathOperator{\AS}{AS}
\DeclareMathOperator{\Aut}{Aut}
\DeclareMathOperator{\Br}{Br}
\DeclareMathOperator{\BT}{BT}
\DeclareMathOperator{\Char}{char}
\DeclareMathOperator{\chr}{char}
\def\char{\text{char}}
\DeclareMathOperator{\Cl}{Cl}
\DeclareMathOperator{\coker}{coker}
\DeclareMathOperator{\cris}{cris}
\DeclareMathOperator{\cts}{cts}
\DeclareMathOperator{\Disc}{Disc}
\DeclareMathOperator{\disc}{disc}
\DeclareMathOperator{\Div}{Div}
\DeclareMathOperator{\Pdiv}{Prin}
\DeclareMathOperator{\ddiv}{div}
\DeclareMathOperator{\et}{\acute{e}t}
\DeclareMathOperator{\End}{End}
\DeclareMathOperator{\Ext}{Ext}
\DeclareMathOperator{\Frac}{Frac}
\DeclareMathOperator{\Frob}{Frob}
\DeclareMathOperator{\Gal}{Gal}
\DeclareMathOperator{\GL}{GL}
\DeclareMathOperator{\GO}{GO}
\DeclareMathOperator{\GSp}{GSp}
\DeclareMathOperator{\Hom}{Hom}
\DeclareMathOperator{\im}{im}
\DeclareMathOperator{\id}{id}
\DeclareMathOperator{\Ind}{Ind}
\DeclareMathOperator{\Isom}{Isom}
\DeclareMathOperator{\Jac}{Jac}
\DeclareMathOperator{\Mon}{Mon}
\DeclareMathOperator{\mult}{mult}
\DeclareMathOperator{\new}{new}
\DeclareMathOperator{\Nm}{Nm}
\DeclareMathOperator{\Norm}{Nm}
\DeclareMathOperator{\NS}{NS}
\DeclareMathOperator{\old}{old}
\DeclareMathOperator{\ord}{ord}
\DeclareMathOperator{\Out}{Out}
\DeclareMathOperator{\pgcd}{pgcd}
\DeclareMathOperator{\Pic}{Pic}
\DeclareMathOperator{\PGL}{PGL}
\DeclareMathOperator{\PSL}{PSL}
\DeclareMathOperator{\PU}{PU}
\DeclareMathOperator{\proj}{proj}
\DeclareMathOperator{\Proj}{Proj}
\DeclareMathOperator{\res}{res}
\DeclareMathOperator{\Res}{Res}
\DeclareMathOperator{\rig}{rig}
\DeclareMathOperator{\sgn}{sgn}
\DeclareMathOperator{\SL}{SL}
\DeclareMathOperator{\SO}{SO}
\DeclareMathOperator{\sol}{sol}
\DeclareMathOperator{\Sp}{Sp}
\DeclareMathOperator{\ssim}{ss}
\DeclareMathOperator{\Spec}{Spec}
\DeclareMathOperator{\Span}{Span}
\DeclareMathOperator{\Spf}{Spf}
\DeclareMathOperator{\SU}{SU}
\DeclareMathOperator{\Sym}{Sym}
\DeclareMathOperator{\Tor}{Tor}
\DeclareMathOperator{\tr}{tr}
\DeclareMathOperator{\Tr}{Tr}
\DeclareMathOperator{\tors}{tors}
\DeclareMathOperator{\rk}{rk}
\DeclareMathOperator{\Sel}{Sel}
\DeclareMathOperator{\Twist}{Twist}
\DeclareMathOperator{\Supp}{Supp}
% Default pretty print macros:
\def\l{\ell}
\def\a{\mathfrak{a}}
\def\b{\mathfrak{b}}
\def\f{\mathfrak{f}}
\def\m{\mathfrak{m}}
\def\n{\mathfrak{n}}
\def\p{\mathfrak{p}}
\def\q{\mathfrak{q}}
\def\r{\mathfrak{r}}
\def\lf{\mathfrak{l}}
\def\A{\mathbb{A}}
\def\C{\mathbb{C}}
\def\F{\mathbb{F}}
\def\G{\mathbb{G}}
\def\H{\mathcal{H}}
\def\L{\mathcal{L}}
\def\N{\mathbb{N}}
\def\O{\mathcal{O}}
\def\P{\mathbb{P}}
\def\Q{\mathbb{Q}}
\def\R{\mathbb{R}}
\def\T{\mathbb{T}}
\def\Z{\mathbb{Z}}

% Overline macros:
\def\abar{\overline{a}}
\def\bbar{\overline{b}}
\def\fbar{\overline{f}}
\def\kbar{\overline{k}}
\def\xbar{\overline{x}}
\def\Cbar{\overline{C}}
\def\Ebar{\overline{E}}
\def\Fbar{\overline{\F}}
\def\Gbar{\overline{G}}
\def\Kbar{\overline{K}}
\def\Qbar{\overline{\Q}}
\def\xbar{\overline{x}}
\def\Xbar{\overline{X}}
\def\ybar{\overline{y}}
\def\Ybar{\overline{Y}}
\def\Zbar{\overline{Z}}
\def\phibar{\overline{\varphi}}

% Hat macros:
\def\Phat{\widehat{P}}
\def\Xhat{\widehat{X}}
\def\Yhat{\widehat{Y}}
\def\Zhat{\widehat{Z}}

% Blackboard bold macros:
\def\AA{\mathbb{A}}
\def\CC{\mathbb{C}}
\def\FF{\mathbb{F}}
\def\HH{\mathbb{H}}
\def\NN{\mathbb{N}}
\def\PP{\mathbb{P}}
\def\QQ{\mathbb{Q}}
\def\RR{\mathbb{R}}
\def\TT{\mathbb{T}}
\def\ZZ{\mathbb{Z}}

% Bold macros:
\def\Ab{\mathbf{A}}
\def\cb{\mathbf{c}}
\def\Cb{\mathbf{C}}
\def\Db{\mathbf{D}}
\def\Fb{\mathbf{F}}
\def\Gb{\mathbf{G}}
\def\Hb{\mathbf{H}}
\def\Nb{\mathbf{N}}
\def\Pb{\mathbf{P}}
\def\Qb{\mathbf{Q}}
\def\Rb{\mathbf{R}}
\def\Sb{\mathbf{S}}
\def\ub{\mathbf{u}}
\def\Zb{\mathbf{Z}}

% Calligraphic macros:
\def\Ac{\mathcal{A}}
\def\Bc{\mathcal{B}}
\def\Cc{\mathcal{C}}
\def\Dc{\mathcal{D}}
\def\Ec{\mathcal{E}}
\def\Fc{\mathcal{F}}
\def\Hc{\mathcal{H}}
\def\Mc{\mathcal{M}}
\def\Nc{\mathcal{N}}
\def\Oc{\mathcal{O}}
\def\Pc{\mathcal{P}}
\def\Rc{\mathcal{R}}
\def\Sc{\mathcal{S}}
\def\Tc{\mathcal{T}}
\def\Uc{\mathcal{U}}
\def\Xc{\mathcal{X}}
\def\Yc{\mathcal{Y}}

%Fraktur macros:
\def\af{\mathfrak{a}}
\def\nf{\mathfrak{n}}
\def\Nf{\mathfrak{N}}
\def\pf{\mathfrak{p}}
\def\Pf{\mathfrak{P}}
\def\qf{\mathfrak{q}}
\def\Qf{\mathfrak{Q}}
\def\rf{\mathfrak{r}}
\def\Rf{\mathfrak{R}}

% Pretty / script macros:
\def\Ap{\mathscr{A}}
\def\Bp{\mathscr{B}}
\def\Cp{\mathscr{C}}
\def\Dp{\mathscr{D}}
\def\Ep{\mathscr{E}}
\def\Fp{\mathscr{F}}
\def\Hp{\mathscr{H}}
\def\Mp{\mathscr{M}}
\def\Np{\mathscr{N}}
\def\Op{\mathscr{O}}
\def\Pp{\mathscr{P}}
\def\Rp{\mathscr{R}}
\def\Tp{\mathscr{T}}
\def\Xp{\mathscr{X}}
\def\Yp{\mathscr{Y}}
\def\Zp{\mathscr{Z}}

% Sans serif macros:
\def\As{\mathsf{A}}
\def\Cs{\mathsf{C}}
\def\Fs{\mathsf{F}}
\def\Hs{\mathsf{H}}
\def\Qs{\mathsf{Q}}
\def\Ms{\mathsf{M}}
\def\Ss{\mathsf{S}}
\def\Vs{\mathsf{V}}
\def\Xs{\mathsf{X}}
\def\Ys{\mathsf{Y}}


\newtheorem{proposition}{Proposition}

\newcommand{\adrian}[1]{{\color{blue} \sf
    $\spadesuit\spadesuit\spadesuit$ Adrian: [#1]}}
    
    
\newcommand\makenode[1]{node [text width={max(0.3*width("#1"),3cm)}] {#1}}


\title{A Blockchain Solution for\\ Digital Identity Management for Refugees}

\author[Adrian Dina]{Dr. B. Adrian Dina}

\date[Tel Aviv, 03/July/2024]{Udacity:\\ Blockchain Developer \\ 20 July 2024}





\begin{document}

\begin{frame}[plain]
  \titlepage
\end{frame}




\begin{frame}{Project Overview}
\alert{Introduction}: Refugees often face significant challenges in 
\vspace{-0.3cm}
\begin{itemize}
\item proving their identity and 
\item integrating into new communities 
\end{itemize}
\vspace{-0.2cm}
due to a lack of reliable identification documents. \newline

\alert{We propose}: A blockchain-based digital identity management system that provide a secure, immutable, and portable solution to address these issues.
\end{frame}



\begin{frame}{Traditional Approaches}

\alert{The traditional approach}: We consider the schematic representation of refugee reception in \underline{Germany}; we restrict on the first two steps of the actual registration process: 
\begin{enumerate}
\item \underline{Arrival and First Contact}:  
	\begin{itemize}
		\item After entering German territory, refugees must make contact with the authorities; this can be done at border posts, police stations, or directly at reception centers.
		\item Afterwards, refugees are transported to an initial reception center called ''Erstaufnahmeeinrichtung''.
	\end{itemize}
\item \underline{Reception and Identification}: 
	\begin{itemize}
		\item At the reception center, personal data is recorded, including fingerprints and photographs.
		\item After verifying the refugee's identity, they undergo a medical examination.
	\end{itemize}
\end{enumerate}

\end{frame}





\begin{frame}{Some Challenges with Traditional Approaches}

Some challenges regarding the traditional approach in the second step, \underline{Reception and Identification}: 
\begin{itemize}
\item Individuals arrive without proper documentation; some may have \underline{lost their documents}, while \underline{others may intentionally} \underline{discard them} to obscure their origins.
\item This can complicate efforts to ascertain the true identity and background of applicants, which is crucial for security and asylum determination purposes.
\end{itemize}
\end{frame}



\begin{frame}{Our Blockchain Solution $(1)$}
\alert{New Approach}: We propose a blockchain-based digital identity management system (BBDIMS) that can provide a secure, immutable and portable solution to address these issues.

\underline{Arrival and First Contact}:
\begin{itemize}
		\item Upon first contact, authorities use a mobile app connected to the blockchain system to create a preliminary digital identity for the refugee; a unique identifier (given by a hashed version of initial biometric data) is generated and recorded on the blockchain, ensuring that the refugee's identity can be tracked securely from the outset.
		%\item During transport, authorities can update the refugee's digital identity with location data, ensuring a transparent and immutable record of their journey.
\end{itemize}

\end{frame}




\begin{frame}{Our Blockchain Solution $(2)$}
\underline{Reception and Identification}:
\begin{itemize}
		\item Upon arrival at the reception center, detailed biometric data (fingerprints, photographs) are securely recorded using the blockchain system; this data is hashed and stored on-chain, with actual biometric files stored off-chain in a secure, distributed file system (e.g., IPFS). The blockchain ledger records the hash references.
		\item Smart contracts manage the identity verification process, ensuring that data is immutable and only accessible to authorized entities.
\end{itemize}

\end{frame}



\begin{frame}{Our Blockchain Solution $(3)$}
\underline{Reception and Identification}:
\begin{itemize}
		\item The blockchain system verifies the refugee's identity by cross-referencing the newly recorded biometric data with existing records on the blockchain, ensuring no duplication or fraud.
		\item Once identity is verified, a smart contract triggers the next steps, including medical examination; medical examination results are securely recorded and linked to the refugee's digital identity on the blockchain. This ensures that medical records are tamper-proof and can be accessed by authorized medical personnel only.
\end{itemize}

\end{frame}








\begin{frame}{Our Blockchain Solution $(4)$}
Some \underline{challenges} related to our steps of consideration:
\begin{itemize}
	\item Ensuring immediate and reliable access to the blockchain system for all border posts and police stations, which may have varying levels of technology infrastructure.
	\item Ensuring the accurate and secure collection of biometric data in a potentially high-stress environment.
	\item Managing potential duplicates and ensuring accurate identity verification across different locations and systems.
	\item Balancing transparency and accessibility of medical records with the need to maintain data privacy and comply with regulations.
\end{itemize}
\end{frame}




\begin{frame}{Existing Solutions $(1)$}
%Let's focus on existing solution(s) and analyze the following 
\alert{Question}: What are the (major) problems in the current system(s)?\newline

\underline{Arrival and First Contact}: 
\vspace{-0.3cm}
	\begin{itemize}
		\item Inefficiencies and Delays; 
		%manual processes and paperwork at border posts and police stations lead to delays in registering refugees, and a high influx of refugees can overwhelm the existing infrastructure, causing long wait times.
		\item Lack of Immediate Tracking; 
		%difficulty in tracking refugees from their point of entry to the reception centers due to disparate and unconnected systems.
		\item Data Fragmentation; 
		%initial data collected might not be immediately shared with the reception centers, leading to incomplete records and potential loss of information.
	\end{itemize}
\end{frame}



\begin{frame}{Existing Solutions $(2)$}
\underline{Reception and Identification}:
\vspace{-0.3cm}
\begin{itemize}
	%\item \underline{Inconsistent Communication}; between border authorities and reception centers can lead to issues in managing the arrival and processing of refugees.
	\item Recording Personal Data; 
	%reliance on manual data entry increases the risk of errors and inconsistencies in refugee records. Further, paper-based records can be lost or damaged, leading to loss of critical information. Without a unified system, there is a risk of creating duplicate records for the same individual, complicating the verification process.
	\item Identity Verification and Medical Examination; 
	%sensitive personal and medical information is at risk of being mishandled or accessed by unauthorized personnel, raising privacy and security concerns. 
\end{itemize}		
\end{frame}


\begin{comment}

\begin{frame}{BBDIMS Solution $(1)$}
\alert{Question}: Will decentralized features, such as transparency, access, censorship resistance, and resilience, provide value in our use case?\newline

\underline{Transparency}:
\begin{itemize}
\item  Trust Building; 
%transparent processes and records build trust among refugees, authorities, and service providers. Refugees can see and understand how their data is used and verified. 
\end{itemize}
\underline{Access}:
\begin{itemize}
\item  Portability; 
%decentralized identity systems allow refugees to access their digital identities from anywhere, without reliance on a single centralized authority or database.
\item Inclusion; 
%ensures that refugees, regardless of location, have equal access to their identity records and related services, promoting greater social and economic integration.
\end{itemize}
\end{frame}




\begin{frame}{BBDIMS Solution $(2)$}
\underline{Censorship resistance}:
\begin{itemize}
\item  Autonomy; 
%decentralized systems prevent any single entity from unilaterally altering or suppressing identity records, protecting refugees from potential misuse of power.
\item Data Integrity; 
%ensures that once data is entered into the system, it cannot be censored or manipulated, preserving the integrity of the refugees' identities and records.
\end{itemize}
\underline{Resilience}:
\begin{itemize}
\item  Fault Tolerance; 
%decentralized systems are more resilient to failures and attacks. Even if some nodes go offline, the system remains operational, ensuring continuous access to identity records.
\item Disaster Recovery; 
%In case of localized disasters or infrastructure failures, decentralized systems can continue to function, ensuring that refugees' identity data is not lost and services can be maintained.
\end{itemize}
\end{frame}


\end{comment}





\begin{frame}{Blockchain: Hyperledger Fabric}
\alert{Blockchain}: \href{https://www.hyperledger.org/projects/fabric}{Hyperledger Fabric (HF).} \newline

\alert{Consensus Mechanisms}: Practical Byzantine Fault Tolerance (PBFT) and Raft. \newline

Relevant parameters for our use case:
\begin{itemize}
	\item Decentralization; medium - suitable for controlled environments.
	\item Performance; high - over 2000 tps.
	\item Transaction Throughput; very high, suitable for high-volume applications.
	\item Scalability; high - supports growing usage efficiently.
	\end{itemize}
\end{frame}



\begin{frame}{Blockchain: Hyperledger Fabric}
Relevant parameters for our use case:
\begin{itemize}
	\item Privacy and Security; high - robust privacy features and access controls.
	\item Integration and Flexibility; high - modular and flexible architecture.
	\item Compliance with Regulations; high - strong support for data protection regulations.
	\item Ease of Implementation;  moderate - requires moderate expertise for deployment.
	\item Cost Efficiency; moderate - higher setup and maintenance costs.
	\item User Experience; high - user-friendly interfaces and systems.
\end{itemize}
\end{frame}




\begin{frame}{Estimated Number of Transactions}
\alert{Question}: How fast the system needs to process transactions to meet demands (and still work well) as usage grows? \newline
\begin{table}[h!]
\label{estimation1}
\resizebox{1\textwidth}{!}{%
\begin{tabular}{|c|c|c|}
\hline
Activity & Frequency per Refugee & Total Transactions (Annually)\\
\hline
\hline
 Initial Registration & 1 & \alert{100,000} \\
\hline
 Identity Verification & 10 & 1,000,000 \\
\hline
Data Updates &5 & 500,000 \\
\hline
 Access Logs & 20 & 2,000,000 \\
\hline
\hline
 Total Annual Transactions & - & 3,600,000\\
 \hline
 Total Monthly Transactions & - & 300,000 \\
 \hline
 Total Daily Transactions & -  & 9,860 \\
  \hline
 Transactions per Second (tps) & -  & 0.114\\
\hline
\end{tabular}}
\caption{System req. for an estimated number of refugees of 100,000 p.y.}
\end{table}
\end{frame}



\begin{comment}
\begin{table}[h!]
\label{estimation2}
\begin{tabular}{|c|c|}
\hline
Capability & Value  \\
\hline
\hline
 Throughput & > 2000 tps \\
 \hline
 Scalability & high \\
 \hline
 Latency & low \\
\hline
\end{tabular}
\caption{Hyperledger Fabric Capabilities.}
\end{table}
\end{comment}

\begin{frame}{Hyperledger Fabric Capabilities}
\begin{table}[h!]
\label{estimation3}
\resizebox{1\textwidth}{!}{%
\begin{tabular}{|c|c|c|}
\hline
Parameter & Requirement &  Hyperledger Fabric Capability \\
\hline
\hline
 Estimated Daily Transactions & 9,860 & high \\
 \hline
 Estimated tps & 0.114 & > 2000 \\
 \hline
 Scalability and Performance & adequate & high \\
\hline
\end{tabular}}
\end{table}
\alert{Conclusion}: 
\begin{itemize}
\item Given the extremely high transaction throughput of Hyperledger Fabric, the system can theoretically handle up to: \[ 1,728,000,000 \] refugees per day (which is unrealistic for real-world scenarios). 
\item It shows that Hyperledger Fabric's capacity far exceeds typical and even extreme refugee numbers, ensuring that the system will not collapse under normal or even high-demand conditions.
\end{itemize}
\end{frame}






\begin{frame}{Alternative blockchains}
We considered the following (alternative) chains:
\vspace{-0.3cm}
\begin{table}[h!]
\label{estimation3}
\resizebox{1\textwidth}{!}{%
\begin{tabular}{|c|c|c|c|}
\hline
Criteria & \alert{Hyperledger Fabric} &  Quorum & Corda  \\
\hline
\hline
 Performance & high & high & medium \\
 \hline
 Transaction Throughput in tps & > 200 & > 2000 & 170\\
 \hline
 Scalability & high & high & medium \\
  \hline
 Privacy and Security & high & high & very high \\
  \hline
 Integration and Flexibility & high & high & high \\
 \hline
  Compliance with Regulations & high & high & high \\
  \hline
  Ease of Implementation & moderate & moderate & moderate \\
  \hline
Cost Efficiency & moderate & moderate &  high \\
 \hline
 User Experience & high & high & moderate \\
   \hline
Consensus Approach & Practical Byzantine Fault Tolerance, Raft & Raft, Istanbul BFT  & Notary (Raft, BFT) \\
\hline
\end{tabular}}
\end{table}
\alert{Summary}: While Quorum and Corda have their strengths in privacy, security, and specific enterprise applications, Hyperledger Fabric stands out due to its:
\begin{itemize}
\item High transaction throughput and performance;
\item Modular and scalable architecture;
\item Strong compliance with regulatory standards.
\end{itemize}
\end{frame}





\begin{frame}{Technical Implementation $(1)$}
Let's focus on the technical implementation. \newline 


\alert{Overview}: 
\begin{itemize}
\item The high-level architecture will consist of \underline{multiple layers} and \underline{components}, with specific functions residing on the blockchain to ensure security, transparency, and efficiency. 
\item The architecture will also \underline{include off-chain components} for data storage and processing that do not require the blockchain's immutability and decentralization.
\end{itemize}
\end{frame}




\begin{frame}{Technical Implementation $(2)$}
\alert{On-Chain Components}:
%\vspace{-0.2cm}
\begin{itemize}
\item Digital Identity Management:
	\begin{itemize}
		\item Identity Creation and Registration; 
		%store unique identifiers (e.g., hashed biometrics, personal data) for each refugee. Use smart contracts to automate and secure the registration process.
		\item Identity Verification; 
		%smart contracts to verify identities during access to services, and public key infrastructure (PKI) for secure identity authentication.
	\end{itemize}
\item Access Control and Permissions:
	\begin{itemize}
		\item Smart Contracts for Access Control; 
		%manage permissions for who can read/write/update data on the blockchain, and ensure only authorized entities (e.g., government agencies, NGOs) can access or modify sensitive information
	\end{itemize}
\item Transaction Logging:
	\begin{itemize}
		\item Immutable Logs; 
		%record all transactions (identity creation, updates, verifications) to provide an audit trail, and ensures transparency and traceability for all actions on the system.
	\end{itemize}
\end{itemize}

\end{frame}



\begin{frame}{Technical Implementation $(3)$}
\alert{Off-Chain Components}:
\begin{itemize}
\item Data Storage:
	\begin{itemize}
		\item Secure Off-Chain Storage; 
		%store large and sensitive personal data (detailed biometric data, medical records) in a secure, distributed file system (e.g., IPFS, cloud storage); store only the hash references of these data on the blockchain to ensure integrity and immutability. 
	\end{itemize}
\item Application Layer:
	\begin{itemize}
		\item User Interfaces; 
		%Web and mobile applications for refugees to manage their identities, and Interfaces for authorities and service providers to verify identities and update records.
		\item API Gateway; 
		%APIs to facilitate communication between the blockchain and external systems (e.g., government databases, service provider systems).
	\end{itemize}
\item Middleware: 
	\begin{itemize}
		\item Integration Layer; 
		%middleware to connect on-chain smart contracts with off-chain data storage (external systems). Data synchronization to ensure consistency between on/off-chain.
	\end{itemize}
\end{itemize}

\end{frame}



\begin{frame}{Architecture Diagram $(1)$}
\vspace{0.1cm}
\resizebox{0.9\textwidth}{!}{%
\begin{tikzpicture}[
node distance=1.5cm]

% Nodes
\node (arrival) [startstop] {Refugee Arrival};
\node (contact) [process, below of=arrival] {\begin{tabular}{c}  Initial Contact\\ (Authorities) \end{tabular}};
\node (registration) [process, below of=contact] {\begin{tabular}{c} Identity Registration\\ (Biometrics, Personal Data) \end{tabular}};



% Arrows
\draw [arrow] (arrival) -- (contact);
\draw [arrow] (contact) -- (registration);

\end{tikzpicture}}
\end{frame}



\begin{frame}{Architecture Diagram $(2)$}
\vspace{0.6cm}
\resizebox{0.9\textwidth}{!}{%
\begin{tikzpicture}[
node distance=3.0cm]

% Nodes

\node (onchain) [onchain] {\begin{tabular}{c} \underline{On-Chain}: Digital Identity;\\- Create \& Store ID Hashes\\- Issue Unique Identifier\\- Smart Contracts for Access \end{tabular}};  



\node (offchain) [offchain, below of=onchain] {\begin{tabular}{c}  \underline{Off-Chain}: Secure Data Storage;\\ - Store Detailed Biometrics \& Medical Data\\ - Store Large Personal Documents\\ - Hash References Stored On-Chain \end{tabular}};


% Arrows
\draw [arrow] (onchain) -- (offchain);
%\draw [arrow] (offchain) -- (access);
%\draw [arrow] (access) -- (ui);
%\draw [arrow] (ui) -- (middleware);

% Labels
%\draw [decorate,decoration={brace,amplitude=10pt,mirror,raise=5pt},yshift=0pt] (6,-1.5) -- (6,-9.5) node [black,midway,xshift=0.8cm] {On-Chain};
%\draw [decorate,decoration={brace,amplitude=10pt,raise=5pt},yshift=0pt] (6,-11) -- (6,-15) node [black,midway,xshift=0.8cm] {Off-Chain};
\end{tikzpicture}}
\end{frame}




\begin{frame}{Architecture Diagram $(3)$}
%\vspace{0.2cm}
\resizebox{0.9\textwidth}{!}{%
\begin{tikzpicture}[
node distance=1.8cm]

% Nodes

\node (access) [onchain] {\begin{tabular}{c} Access \& Verification;\\ - Services (Healthcare, etc.)\\ - Verify ID via Blockchain \end{tabular}};

\node (ui) [process, below of=access] {\begin{tabular}{c} User Interface (Web/Mobile);\\ - Refugee Identity Management\\ - Authorities' Access Control \end{tabular}};
\node (middleware) [process, below of=ui] {\begin{tabular}{c} Middleware/API Gateway;\\ - Integrates On-Chain and Off-Chain\\ - Interfaces with External Systems \end{tabular}};


% Arrows

%\draw [arrow] (offchain) -- (access);
\draw [arrow] (access) -- (ui);
\draw [arrow] (ui) -- (middleware);

% Labels
%\draw [decorate,decoration={brace,amplitude=10pt,mirror,raise=5pt},yshift=0pt] (6,-1.5) -- (6,-9.5) node [black,midway,xshift=0.8cm] {On-Chain};
%\draw [decorate,decoration={brace,amplitude=10pt,raise=5pt},yshift=0pt] (6,-11) -- (6,-15) node [black,midway,xshift=0.8cm] {Off-Chain};
\end{tikzpicture}}
\end{frame}









\begin{frame}{Architecture Challenges $(1)$}
We present the two major challenges for our solution.\newline 


\underline{Data Security and Privacy}:
\begin{itemize}
\item Challenge: \underline{Protecting} sensitive personal and biometric data from unauthorized access and breaches.
\item Addressing the Challenge:
	\begin{itemize}
		\item Encryption; 
		%implement strong encryption for both data at rest and in transit.
		\item Access Controls; 
		%use fine-grained access controls and smart contracts to manage permissions.
		\item Regular Audits; 
		%conduct regular security audits and vulnerability assessments to identify and mitigate risks.	
	\end{itemize}
\end{itemize}
\end{frame}


\begin{frame}{Architecture Challenges $(2)$}
\underline{Adoption Incentives}:
\begin{itemize}
\item Challenge: Encouraging stakeholders (government agencies, NGOs, refugees) to \underline{adopt and trust the new system}.
\item Addressing the Challenge:
	\begin{itemize}
		\item Pilot Programs; 
		%start with pilot programs to demonstrate the system's benefits and build trust.
		\item Stakeholder Engagement; 
		%involve key stakeholders early in the development process to align the system with their needs and gain their support.
		\item Incentives; 
		%provide incentives for early adopters, such as improved service access for refugees or streamlined operations for authorities.	
	\end{itemize}
\end{itemize}
\end{frame}




\begin{frame}{Alternatives to Complement Blockchain Aspects}
\alert{Question}: What alternatives could complement the blockchain aspects?
\begin{enumerate}
\item \underline{Distributed Storage Solutions}: 
	\begin{itemize}
		\item Example; InterPlanetary File System (IPFS).
		\item Function; store large files and sensitive data off-chain while using the blockchain to store hash references, ensuring data integrity and availability without overloading the blockchain.
	\end{itemize}
\item \underline{Federated Identity Systems}:
	\begin{itemize}
		\item Example; Self-Sovereign Identity (SSI) Frameworks.
		\item Function; allow refugees to manage their identities and control access to their personal data, enhancing privacy and user autonomy.
	\end{itemize}
\item \underline{Cloud-Based Infrastructure}:
	\begin{itemize}
		\item Example; Amazon Web Services (AWS), Microsoft Azure.
		\item Function; provide scalable and compliant infrastructure for off-chain components, such as secure data storage and API gateways.
	\end{itemize}
\end{enumerate}
\end{frame}



\begin{frame}
Literature:
\begin{itemize}
\item \url{https://www.hyperledger.org/projects/fabric}
\item \url{https://goquorum.readthedocs.io}
\item \url{https://corda.net}
\item Scholar GPT 
\end{itemize}
\end{frame}


\begin{frame}
\centering \Large
  \emph{Thank you for listening!}
\end{frame}


\end{document}

